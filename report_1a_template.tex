%\documentclass[12pt,oneside]{jarticle} % 一般的なドキュメントクラス
\documentclass[12pt,oneside]{ise-thesis} % 片面印刷の場合
%\documentclass[12pt,twoside]{ise-thesis} % 両面印刷の場合


\newcommand{\figref}[1]{図~\ref{#1}}
\newcommand{\tabref}[1]{表~\ref{#1}}
\newcommand{\equref}[1]{式~(\ref{#1})}
%\setlength\textfloatsep{100pt}
\setlength\intextsep{8pt} % 本文中の図の余白
\setlength\abovecaptionskip{5pt} % 図とキャプションの間の余白
\usepackage{comment}
\usepackage{here}
\usepackage{multirow}
\usepackage{mediabb}


% 行間調整
\renewcommand{\baselinestretch}{1.15}
\sectionstretch{1.0}		% セクション見出し前後の行間
\liststretch{1.0}		% 箇条書き環境の行間



\begin{document}

\begin{flushleft}
  MM工学研究Ia 中間報告資料
\end{flushleft}
\begin{center}
\vspace*{12pt}
{\Large ここに研究テーマのタイトルを記入}
\vspace{12pt} \\
\begin{tabular}{c}
氏名 (学籍番号: ○○○○○○)
\end{tabular} \vspace{3pt} \\
\today \vspace{12pt} \\
\end{center}


%第1章:研究背景
% 第1章
\section*{1. 研究背景}

ここに研究背景などを書いてください.

%第2章:研究領域の概要
% 第2章
\section*{2. 研究領域の概要}

ここに調査対象分野の概要などを書いてください.

%第3章:関連研究 (中間進捗)
% 第3章
\section*{3. 関連研究 (中間進捗)}

ここに関連研究などを書いてください.

%第4章:今後の研究予定
% 第4章
\section*{4. 今後の研究予定}

ここに今後の研究動向調査の予定などを書いてください.
\newline

参考文献は卒論の時と同じ\cite{yan2018spatial}

% 参考文献
% 日本語BibTeX (pbibtex/jbibtex) を使う場合
\bibliographystyle{ieice}
%\bibliography{./ref} % default
\bibliography{ref} % change "./ref"(default) to "ref"

\end{document}